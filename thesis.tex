\documentclass[12pt,draft,letterpaper]{report}

%% The following makes chapters and sections, but not subsections,
%% appear in the TOC (table of contents). Increase to 2 or 3 to
%% make subsections or subsubsections appear, respectively. It seems
%% to be usual to use the "1" setting, however.
\setcounter{tocdepth}{1}

%% Sectional units up to subsubsections are numbered. To number
%% subsections, but not subsubsections, decrease this counter to 2.
\setcounter{secnumdepth}{3}

% Setting a gap between page number and text block


%% Page layout (customized to letter paper and NYU requirements):
\setlength{\oddsidemargin}{.6in}
\setlength{\textwidth}{5.8in}
\setlength{\topmargin}{0.5in}
\setlength{\headheight}{0in}
\setlength{\headsep}{0in}
\setlength{\textheight}{8.3in}
\setlength{\footskip}{.5in}

%% Use the following commands, if desired, during production.
%% Comment them out for final version.
%\usepackage{layout} % defines the \layout command, see below
%\setlength{\hoffset}{-.75in} % creates a large right margin for notes and \showlabels

%% Controls spacing between lines (\doublespacing, \onehalfspacing, etc.):
\usepackage{setspace}

%
%% \usepackage{amsmath}
%% \usepackage{amssymb}
\usepackage{xspace}
\usepackage{algorithmic}
\usepackage{algorithm}
\usepackage{microtype}
\usepackage{subfigure}
\usepackage{color}
\usepackage{url}
\usepackage{lipsum}
\usepackage{fancyhdr}
% \newfloat{algorithm}{t}{lop}

\pagestyle{fancy}
\fancyhf{}
\renewcommand{\headrulewidth}{0pt}
\fancyhead[LE]{}
\fancyhead[RO]{}
\fancyhead[RE]{}
\fancyhead[LO]{}
\fancyfoot[C]{}
\rhead{\thepage}

\fancypagestyle{plain}{%
\fancyhf{}
\rhead{\thepage}
}

\setlength{\headheight}{20pt} 

%% Use the line below for official NYU version, which requires
%% double line spacing. For all other uses, this is unnecessary,
%% so the line can be commented out.
\onehalfspacing % requires package setspace, invoked above

%% Each of the following lines defines the \com command, which produces
%% a comment (notes for yourself, for instance) in the output file.
%% Example:    \com{this will appear as a comment in the output}
%% Choose (uncomment) only one of the three forms:
%\newcommand{\com}[1]{[/// {#1} ///]}       % between [/// and ///].
\newcommand{\com}[1]{\marginpar{\tiny #1}} % as (tiny) margin notes
%\newcommand{\com}[1]{}                     % suppress all comments.


\begin{document}

\pagenumbering{roman}




%%%%%%%%%%%%%% Abstract %%%%%%%%%%%%%%%%%

\addcontentsline{toc}{section}{Abstract}
\input{abstract\abstract}
\newpage

%%%% Table of Contents %%%%%%%%%%%%
\tableofcontents
% \clearpage
% \pagestyle{headings}

%%%%% List of Figures %%%%%%%%%%%%%
%% Comment out the following two lines if your thesis does not
%% contain any figures. The list of figures contains only
%% those figures included withing the "figure" environment.
\listoffigures\addcontentsline{toc}{section}{\listfigurename}
\newpage

%%%%% List of Tables %%%%%%%%%%%%%
%% Comment out the following two lines if your thesis does not
%% contain any tables. The list of tables contains only
%% those tables included withing the "table" environment.
\listoftables\addcontentsline{toc}{section}{\listtablename}
\newpage

%%%%% Body of thesis starts %%%%%%%%%%%%
\pagenumbering{arabic} % switches page numbering to arabic: 1, 2, 3, etc.

\input{Introduction/introduction}
\input{Conclusion/conclusion}



\end{document}